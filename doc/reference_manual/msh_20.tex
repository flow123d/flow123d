

\parindent = 0pt

\section*{Mesh file format version 2.0}
The mesh file format comes from the GMSH system. 
Following text is copied from the GMSH documentation.\\[0.5em]

{\tt =============== BEGIN OF INSERTED TEXT ===============}\\[0.3em]

Version 2.0 of the {\tt .MSH} file format is Gmsh's new native mesh file
format. It is very similar to the old one (Version 1.0), but is more
general: it contains information about itself and allows to associate an
arbitrary number of integer tags with each element.

The {\tt .MSH} file format, version 2.0, is divided in three sections,
defining the file format ({\tt \$MeshFormat}-{\tt \$EndMeshFormat}), the
nodes ({\tt \$Nodes}-{\tt \$EndNodes}) and the elements
({\tt \$Elements}-{\tt \$EndElements}) in the mesh:

\begin{fileformat}
\$MeshFormat\\
2.0 \vari{file-type} \vari{data-size}\\
\$EndMeshFormat\\
\$Nodes\\
\vari{number-of-nodes}\\
\vari{node-number} \vari{x-coord} \vari{y-coord} \vari{z-coord}\\
\dots\\
\$EndNodes\\
\$Elements\\
\vari{number-of-elements}\\
\vari{elm-number} \vari{elm-type} \vari{number-of-tags} \vari{$<$tags$>$} \vari{node-number-list}\\
\dots\\
\$EndElements
\end{fileformat}

where:
\begin{description}
\item[\vari{file-type}]
is an integer equal to 0 in the ASCII file format.

\item[\vari{data-size}]
is an integer equal to the size of the floating point numbers used in the
file (usually, \vari{data-size} = sizeof(double)).

\item[\vari{number-of-nodes}]
is the number of nodes in the mesh.

\item[\vari{node-number}]
is the number (index) of the \vari{n}-th node in the mesh. Note that the
\vari{node-number}s do not have to be given in a consecutive (or even an
ordered) way.

\item[\vari{x-coord} \vari{y-coord} \vari{z-coord}]
are the floating point values giving the X, Y and Z coordinates of the
\vari{n}-th node.

\item[\vari{number-of-elements}]
is the number of elements in the mesh.

\item[\vari{elm-number}]
is the number (index) of the \vari{n}-th element in the mesh. Note that the
\vari{elm-number}s do not have to be given in a consecutive (or even an
ordered) way.

\item[\vari{elm-type}]
defines the geometrical type of the \vari{n}-th element:

\begin{tabular}{ll}
1 & Line (2 nodes) \\
2 & Triangle (3 nodes) \\
3 & Quadrangle (4 nodes) \\
4 & Tetrahedron (4 nodes) \\
5 & Hexahedron (8 nodes) \\
6 & Prism (6 nodes) \\
7 & Pyramid (5 nodes) \\
8 & Second order line (3 nodes) \\
9 & Second order triangle (6 nodes) \\
11 & Second order tetrahedron (10 nodes) \\
15 & Point (1 node) \\
\end{tabular}

\item[\vari{number-of-tags}]
gives the number of tags for the \vari{n}-th element. By default, Gmsh
generates meshes with two tags and reads files with an arbitrary number of
tags: see below.

\item[\vari{tag}]
is an integer tag associated with the \vari{n}-th element. By default, the
first tag is the number of the physical entity to which the element belongs;
the second is the number of the elementary geometrical entity to which the
element belongs; the third is the number of a mesh partition to which the
element belongs.

\item[\vari{node-number-list}]
is the list of the node numbers of the \vari{n}-th element (separated by
white space, without commas). The ordering of the nodes is given in
Gmsh node ordering; for second order elements, the first order nodes
are given first, followed by the nodes associated with the edges, followed
by the nodes associated with the faces (if any). The ordering of these
additional nodes follows the ordering of the edges/faces given in Gmsh
node ordering. 
\end{description}
 
{\tt =============== END OF INSERTED TEXT ===============}\\[0.5em]

More information about GMSH can be found at its homepage:\\
\leftline{\tt http://www.geuz.org/gmsh/}\\

\subsection*{Comments concerning {\tt 1-2-3-FLOW}:}
\begin{itemize}
  \item Every inconsistency of the file stops the calculation.
    These are:
      \begin{itemize}
        \item Existence of nodes with the same \vari{node-number}.
        \item Existence of elements with the same \vari{elm-number}.
        \item Reference to non-existing node.
        \item Reference to non-existing material (see below).
        \item Difference between \vari{number-of-nodes} and actual number of
          lines in nodes' section.
        \item Difference between \vari{number-of-elements} and actual number of
          lines in elements' section.
      \end{itemize}
  \item By default {\tt 1-2-3-FLOW} uses meshes with \vari{number-of-tags} = 2.
    \begin{description}
    \item[\vari{tag1}] is number of region in which the element lies. 
    \item[\vari{tag2}] is number of material (reference to {\tt
    .MTR} file) in the element.
    \end{description}
  \item Currently, line (\vari{type} = 1), triangle (\vari{type} = 2) and
    tetrahedron (\vari{type} = 4) are the only supported types
    of elements. Existence of an element of different type stops the calculation.
  \item Wherever possible, we use the file extension {\tt .MSH}. It is not
    required, but highly recomended.
\end{itemize}

%%%%%%%%%%%%%%%%%%%%%%%%%%%%%%%%%%%%%%%%%%%%%%%%%%%%%%%%%%%%%%%%%%%%%%%%%%%%%%
% vim: set tw=78 ts=2 sw=2 expandtab nocindent smartindent:
%%%%%%%%%%%%%%%%%%%%%%%%%%%%%%%%%%%%%%%%%%%%%%%%%%%%%%%%%%%%%%%%%%%%%%%%%%%%%%


