% Copyright (C) 2007 Technical University of Liberec.  All rights reserved.
%
% Please make a following refer to Flow123d on your project site if you use the program for any purpose,
% especially for academic research:
% Flow123d, Research Centre: Advanced Remedial Technologies, Technical University of Liberec, Czech Republic
%
% This program is free software; you can redistribute it and/or modify it under the terms
% of the GNU General Public License version 3 as published by the Free Software Foundation.
%
% This program is distributed in the hope that it will be useful, but WITHOUT ANY WARRANTY;
% without even the implied warranty of MERCHANTABILITY or FITNESS FOR A PARTICULAR PURPOSE.
% See the GNU General Public License for more details.
%
% You should have received a copy of the GNU General Public License along with this program; if not,
% write to the Free Software Foundation, Inc., 59 Temple Place - Suite 330, Boston, MA 021110-1307, USA.

%%%%%%%%%%%%%%%%%%%%%%%%%%%%%%%%%%%%%%%%%%%%%%%%%%%%%%%%%%%%%%%%%%%%%%%%%%%%%%
\section*{Boundary conditions file format, version 1.0}
The file is divided in two sections, header and data.
\begin{fileformat}
\$BoundaryFormat\\
  1.0 \vari{file-type} \vari{data-size}\\
\$EndBoundaryFormat\\
\$BoundaryConditions\\
  \vari{number-of-conditions}\\
  \vari{condition-number} \vari{type} \vari{$<$type-specific-data$>$}
  \vari{where} \vari{$<$where-data$>$}
  \vari{number-of-tags} \vari{$<$tags$>$} \vari{[text]}\\
  \dots\\
\$EndBoundaryConditions\\
\end{fileformat}
where
\begin{description}
 \ditem{file-type}{int} --- is equal 0 for the ASCII file format.
 \ditem{data-size}{int} --- the size of the floating point numbers used in
  the file. Usually \vari{data-size} = sizeof(double).
 \ditem{number-of-conditions}{int} --- Number of boundary conditions defined in the
  file.
 \ditem{condition-number}{int} --- is the number (index) of the n-th boundary
  condition. These numbers do not have to be given in a consecutive (or even an
  ordered) way. Each number has to be given only onece, multiple definition
  are treated as inconsistency of the file and cause stopping the
  calculation.
 \ditem{type}{int} --- is type of the boundary condition. See below for
   definitions of the types.
 \ditem{$<$type-specific-data$>$}{} --- format of this list depends on the
   \vari{type}. See below for specification of the \vari{type-specific-data}
   for particular types of the boundary conditions.
 \ditem{where}{int} --- defines the way, how the place for the contidion is
   prescribed. See below for details.
 \ditem{$<$where-data$>$}{} --- format of this list depends on \vari{where}
   and actually defines the place for the condition. See below for details.
 \ditem{number-of-tags}{int} --- number of integer tags of the boundary
  condition. It can be zero.
 \ditem{$<$ tags $>$}{{\vari{number-of-tags}*}int} --- list of tags of the
   boundary condition. Values are
   separated by spaces or tabs. By default we set
   \vari{number-of-tags}=1, where \vari{tag1} defines group of boundary
   conditions, "type of water" in our jargon. This can be used to calculate total fluxes through 
   the boundary group.
 \ditem{[text]}{char[]} --- arbitrary text, description of the fracture, notes,
   etc., up to 256 chars. This is an optional parameter.
\end{description}
\subsection*{Types of boundary conditions and their data}
    \begin{description}
      \ditem{type =}{1} --- Boundary condition of the Dirichlet's type
      \ditem{type =}{2} --- Boundary condition of the Neumann's type
      \ditem{type =}{3} --- Boundary condition of the Newton's type
   \end{description}
   \begin{tabular}{|c|l|l|}
      \hline
      \vari{type} & \vari{type-specific-data} & Description \\
      \hline
      \hline
      1 & \vari{scalar} & Prescribed value of pressure or piez. head \\
      \hline
      2 & \vari{flux} & Prescribed value of flux through the boundary \\
      \hline 
      3 & \vari{scalar} \vari{sigma} & Scalar value and the $\sigma$
      coefficient \\
      \hline
   \end{tabular}

   \vari{scalar}, \vari{flux} and \vari{sigma} are of the {\tt double} type.
\subsection*{Ways of defining the place for the boundary condition}
    \begin{description}
      \ditem{where =}{1} --- Condition on a node
      \ditem{where =}{2} --- Condition on a (generalized) side
      \ditem{where =}{3} --- Condition on side for element with only one
        external side. 
   \end{description}
   \begin{tabular}{|c|l|l|}
     \hline
     \vari{where} & \vari{$<$where-data$>$} & Description \\
     \hline
     \hline
     1 & \vari{node-id} & Node id number, according to {\tt .MSH} file \\
     \hline
     2 & \vari{elm-id} \vari{sid-id} & Elm. id number, local number of side \\
     \hline
     3 & \vari{elm-id} & Elm. id number \\
     \hline
   \end{tabular}

     The variables \vari{node-id}, \vari{elm-id}, \vari{sid-id} are of the
     {\tt int} type. 
   
   %------------------------------------------------------------
\subsection*{Comments concerning {\tt 1-2-3-FLOW}:}
\begin{itemize}
  \item We assume homegemous Neumman's condition as the default one. Therefore
    we do not need to prescribe conditions on the whole boundary.
  \item If the condition is given on the inner edge, it is treated as an error
    and stops calculation.
  \item Any inconsistence in the file stops calculation. (Bad number of
    conditions, multiple definition of condition, reference to non-existing
    node, etc.)
  \item At least one of the conditions has to be of the Dirichlet's or
    Newton's type. This is well-known fact from the theory of the PDE's.
  \item Local numbers of sides for \vari{where} = 2 must be lower than the 
    number of sides of the particular element and greater then or equal to zero.
  \item The element specified for \vari{where} = 3 must have only one external
    side, otherwise the program stops.
\end{itemize}

