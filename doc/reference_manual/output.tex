% Copyright (C) 2007 Technical University of Liberec.  All rights reserved.
%
% Please make a following refer to Flow123d on your project site if you use the program for any purpose,
% especially for academic research:
% Flow123d, Research Centre: Advanced Remedial Technologies, Technical University of Liberec, Czech Republic
%
% This program is free software; you can redistribute it and/or modify it under the terms
% of the GNU General Public License version 3 as published by the Free Software Foundation.
%
% This program is distributed in the hope that it will be useful, but WITHOUT ANY WARRANTY;
% without even the implied warranty of MERCHANTABILITY or FITNESS FOR A PARTICULAR PURPOSE.
% See the GNU General Public License for more details.
%
% You should have received a copy of the GNU General Public License along with this program; if not,
% write to the Free Software Foundation, Inc., 59 Temple Place - Suite 330, Boston, MA 021110-1307, USA.

\label{section_output}

Flow123d support output of scalar and vaector data fields into two formats. First one can use native fromat of program GMSH (usualy with extension \verb'msh')
which contains computational mesh and then various datafields for squence of time levels. For second we support output into XML version of VTK files. These files can be 
viewed and postprocessed by several vizualization softwares. However, our primal goal is to support data transfer into Paraview vizualization software.
See key \hyperlink{KeyOutFormat}{Pos\_format}. 

\subsection{Output data fields of water flow module}
Water flow module provides output of four data fields. 
\begin{description}
 \item[pressure on elements] Pressure head in length units $[L]$ piecewise constant on every element. This field is directly produced by the MH method and thus contains no postprocessing error.
 \item[pressure in nodes] Same pressure head field, but interpolated into $P1$ continuous scalar field. Namely you lost dicontinuities on fractures.
 \item[velocity on elements] Vector field of water flux volume unit per time unit $[L^3 / T]$. For every element we evaluate discrete flux field in barycenter.
 \item[piezometric head on elements] Piezometric head in length units $[L]$ piecewise constant on every element. This is just pressure on element  plus z-coordinate of the barycenter. This field is produced only on demand
 (see key \hyperlink{KeyPiezoHead}{output\_piezo\_head}).
\end{description}

\subsection{Output data fields of transport}
Transport module provides output only for concentrations (in mobile phase) as a field piecewise constant over elements. There is one field for every substance and names of those fields contain 
names of substances given by key \hyperlink{KeySubstanceNames}{Substances}. The physical unit is mass unit over volume unit $[M / L^3]$.



%\subsection{GMSH viewer remarks}

%\subsection{Paraview viewer remarks}

\subsection{Auxiliary output files}

\subsubsection{Profiling information}
On every run we collect some bacis profilling informations. After all computations these data are written into the file
\verb'profiler%y%m%d_%H.%M.%S.out' where \verb'%y', \verb'%m', \verb'%d', \verb'%H', \verb'%M', \verb'%S' are 
two digit numbers representing year, month, day, hour, minute, and second of the program start time.

\subsubsection{Water flux information}
File contains water flow balance, total inflow and outflow over boundary segments. 
Further there is total water income due to sources (if they are present).


\subsubsection{Raw water flow data file}
You can force Flow123d to write raw data about results of MH method. The file format is:
\begin{verbatim}
$FlowField
T=<time>
<number fo elements>
<eid> <pressure> <flux x> <flux y> <flux z> <number of sides> <pressures on sides> <fluxes on sides> 
...
$EndFlowField
\end{verbatim}

where 
\begin{description}
 \item \verb'<time>' --- is simulation time of the raw output.
 \item \verb'<number of elements>' --- is number of elements in mesh, which is same as number of subsequent lines.
 \item \verb'<eid>' --- element id same as in the input mesh.
 \item \verb'<flux x,y,z>' --- components of water flux interpolated to barycenter of the element
 \item \verb'<number of sides>' --- number of sides of the element, infulence number of remaining values
 \item \verb'<pressures on sides>' --- for every side average of the pressure over the side
 \item \verb'<fluxes on sides>' --- for ever side total flux through the side 
\end{description}

