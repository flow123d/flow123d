
\section*{Sources file format, version 1.0}
The file is divided in two sections, header and data.
The extension {\tt .SRC} is highly recomended for files of this type.
\begin{fileformat}
\$SourceFormat\\
  1.0 \vari{file-type} \vari{data-size}\\
\$EndSourceFormat\\
\$Sources\\
  \vari{number-of-sources}\\
  \vari{source-number} \vari{type} \vari{eid} \vari{density}\\
  \dots\\
\$EndSources\\
\end{fileformat}
where
\begin{description}
 \ditem{file-type}{int} --- is equal 0 for the ASCII file format.
 \ditem{data-size}{int} --- the size of the floating point numbers used in
  the file. Usually \vari{data-size} = sizeof(double).
 \ditem{number-of-sources}{int} --- Number of sources defined in the
  file.
 \ditem{source-number}{int} --- is the number (index) of the n-th
  source. These numbers do not have to be given in a consecutive (or even an
  ordered) way. Each number has to be given only onece, multiple definition
  are treated as inconsistency of the file and cause stopping the
  calculation.
 \ditem{type}{int} --- is type of the source. This variable is still unused. 
 \ditem{eid}{int} --- is id-number of the element, where the source lies.
 \ditem{density}{double} --- is the density of the source, in volume of fluid
   per time unit. Possitive values are sources, negative are sinks.
\end{description}
\subsection*{Comments concerning {\tt 1-2-3-FLOW}:}
\begin{itemize}
  \item Every inconsistency or error in the {\tt .SRC} file causes stopping
    the calculation. These are especially:
    \begin{itemize}
      \item Multiple usage of the same \vari{source-number}.
      \item Difference between \vari{number-of-sources} and actual number
        of data lines.
      \item Reference to nonexisting element.
    \end{itemize}
\end{itemize}
