% Copyright (C) 2007 Technical University of Liberec.  All rights reserved.
%
% Please make a following refer to Flow123d on your project site if you use the program for any purpose,
% especially for academic research:
% Flow123d, Research Centre: Advanced Remedial Technologies, Technical University of Liberec, Czech Republic
%
% This program is free software; you can redistribute it and/or modify it under the terms
% of the GNU General Public License version 3 as published by the Free Software Foundation.
%
% This program is distributed in the hope that it will be useful, but WITHOUT ANY WARRANTY;
% without even the implied warranty of MERCHANTABILITY or FITNESS FOR A PARTICULAR PURPOSE.
% See the GNU General Public License for more details.
%
% You should have received a copy of the GNU General Public License along with this program; if not,
% write to the Free Software Foundation, Inc., 59 Temple Place - Suite 330, Boston, MA 021110-1307, USA.

\normalsize

%%%%%%%%%%%%%%%%%        DOCUMENTATION OF GENERAL KINETIC REACTION FOR FUTURE -START           %%%%%%%%%%%%%%%

% \subsection{General Kinetic Reaction}
% \label{sec:kinetic}
% We consider a system of $m$ stoichiometric reactions, each symbolically written as
% \begin{equation} \label{eqn:general_kinetic_reaction}
%   \sum\limits_{i=1}^{n_r} \nu_{rik}\chi_{i} \rightarrow \sum\limits_{i=1}^{n_r} \nu_{pik} \chi_{i},
% \end{equation}
% where 
% \begin{itemize}
%   \item $\nu_{rik}$ \units{}{}{} is the stoichiometric coefficient (number of moles) 
%         for reactant component $i$ in reaction $k$,
%   \item $\nu_{pik}$ \units{}{}{} is the stoichiometric coefficient
%         for product component $i$ in reaction $k$,
%   \item $n_r$ is the number of reaction components (both reactants and products),
%   \item $\chi_{i}$ represents the chemical symbol for the component $i$.
% \end{itemize}
% For the components that are not present in reaction, the stoichiometric coefficients are set
% $\nu_{rik}=0$ or $\nu_{pik}=0$.
% 
% The kinetics temperature dependence is introduce in modified Arrhenius model.
% The production rate of the component $i$ is then modeled as
% \begin{equation} \label{eqn:modified_arrhenius}
%   \frac{\d c_i}{\d t} = M_i \sum\limits_{k=1}^{m}\left( \nu_{pik}-\nu_{rik} \right) 
%   B_k \left(\frac{T}{T_{0k}}\right)^{\alpha_k} \exp\left(-\frac{\Delta E_k}{R_gT}\right)
%   \prod\limits_{j=1}^{n_r}\left(\frac{\rho_j}{M_j}\right)^{\nu_{rjk}},
% \end{equation}
% where
% \begin{itemize}
%   \item $c_i$ \units{1}{-3}{} is concentration of component $i$,
%   \item $M_i$, $M_j$ \units{1}{-3}{} is the molar mass of component $i$, or $j$ respectively,
%   \item $B_k$ \units{}{}{-1} is the collision-frequency factor (or preexponential factor) of reaction $k$,
%               it represents number of all particle collisions per second (not all necessarilly 
%               resulting in reaction),
%   \item $T$ [K] is the current absolute temperature,
%   \item $T_{0k}$ [K] is the reference absolute temperature, at which the number of particle collisions per second
%         is equal $B_k$,
%   \item $\alpha_k$ \units{}{}{} is the temperature exponent of the reaction $k$,
%   \item $\Delta E_k$ $[\textrm{Jmol}^{-1}]$ is the activation energy per mole,
%   \item $R_g = 8.3144$ $[\textrm{Jmol}^{-1}\textrm{K}^{-1}]$ is the universal gas constant,
%   \item $\rho_j$ \units{1}{-3}{} is the density of component $j$.
% \end{itemize}
% 
% To get rid of the unit dependence on the exponent, we divide the equation \eqref{eqn:modified_arrhenius} 
% by liquid density $\rho=\sum_{j=1}^{n_r}\rho_i$ and put $M_i$ under the exponent. Using
% \begin{equation}
%   \prod\limits_{j=1}^{n_r}\left( \frac{\rho_j M_i}{\rho M_j}\right)^{\nu_{rjk}} 
%   = \left(\frac{M_i}{\rho}\right)^{\sum_{j=1}^{n_r}\nu_{rjk}} 
%     \prod\limits_{j=1}^{n_r}\left( \frac{\rho_j}{M_j}\right),
% \end{equation}
% we obtain
% \begin{equation}
%   \frac{\d}{\d t}\left(\frac{c_i}{\rho}\right) = \sum\limits_{k=1}^{m}\left( \nu_{pik}-\nu_{rik} \right) 
%   B_k \left(\frac{T}{T_{0k}}\right)^{\alpha_k} \exp\left(-\frac{\Delta E_k}{R_gT}\right)
%   \left(\frac{M_i}{\rho}\right)^{1-\sum_{j=1}^{n_r}\nu_{rjk}} 
%   \prod\limits_{j=1}^{n_r}\left( \frac{\rho_j M_i}{\rho M_j}\right)^{\nu_{rjk}} 
% %   
% %   = \left(\frac{M_i}{\rho}\right)^{\sum_{j=1}^{n_r}\nu_{rjk}} 
% %     \prod\limits_{j=1}^{n_r}\left( \frac{\rho_j}{M_j}\right),
% \end{equation}

%%%%%%%%%%%%%%%%%          DOCUMENTATION OF GENERAL KINETIC REACTION FOR FUTURE -END           %%%%%%%%%%%%%%%

\subsection{Radioactive Decay}
\label{sec:decay}
The radioactive decay is one of the processes that can be modelled in the reaction term of the transport model.
This process is coupled with the transport using the operator splitting method.
It can run throughout all the phases, including the mobile and immobile phase of the liquid 
and also the sorbed solid phase, as it can be seen in figure \ref{fig:reaction_term}.

The radioactive decay of a parent radionuclide A to a nuclid B
%
\[ A\xrightarrow{k} B, \qquad A\xrightarrow{t_{1/2}} B \]
%
is mathematicaly formulated as a system of first order differential equations
%
\begin{eqnarray} \label{eqn:halflife}
  \frac{\d c_A}{\d\tau} &=& -k c_A, \\
  \frac{\d c_B}{\d\tau} &=& k c_A,
\end{eqnarray}
%
where $k$ is the radioactive decay rate. Usually, the \hyperA{Decay::half-life}{half life} of the parent radionuclide $t_{1/2}$
is known rather than the rate. Relation of these can be derived:
%
% \begin{equation} \label{eqn:halflife}
%   k = - \frac{\ln 2}{t_{1/2}}.
% \end{equation}
\begin{eqnarray*}
    \frac{\d c_A}{\d\tau} &=& -k c_A\\
    \frac{\d c_A}{c_A} &=& -k \d\tau\\
    \int\limits_{c_A^0}^{c_A^0/2}\frac{\d c_A}{c_A} &=& -k\int\limits_{0}^{t_{1/2}} 1\d\tau\\
    \big[ \ln c_A\big]^{c_A^0/2}_{c_A^0} &=& -\big[k\tau\big]^{t_{1/2}}_{0}\\
%     \ln\frac{c_A^0}{2} - \ln c_A^0 &=& - k t_{1/2}\\
%     \ln 2 &=& k t_{1/2}\\
    k &=& \frac{\ln 2}{t_{1/2}}.
\end{eqnarray*}

Let us now suppose a more complex situation. Consider substances (radionuclides) $A_1,\ldots, A_s$ which take 
part in a complex radioactive chain, including branches, e.g.
\begin{center}
\begin{tabular}{rccll}
 $A_1\xrightarrow{k_1}A_2\xrightarrow{k_2}$ & $A_3$ & $ \xrightarrow{k_{34}}$ & $A_4\xrightarrow{k_4}$ & $A_8$ \\
 & $A_3$ & $\xrightarrow{k_{35}} A_5 \xrightarrow{k_{5}}$ & $A_4$ &\\
 & $A_3$ & \multicolumn{2}{c}{$\xrightarrow{k_{36}} A_6 \xrightarrow{k_{6}} A_7 \xrightarrow{k_{7}}$} & $A_8$
\end{tabular}
\end{center}
Now the problem turned into a system of differential equations $\partial_t \vc{c}=\mathbf{D}\vc{c}$ with the following
matrix, generally full and nonsymmetric:
\[
\mathbf{D} = \begin{pmatrix} M_1 &     && \\ 
                                 & M_2 && \\
                                 && \ddots & \\
                  && & M_s \end{pmatrix}
             \begin{pmatrix} -k_1 &k_{21}& \cdots & k_{s1} \\ 
                  k_{12} & -k_2 & \cdots & k_{s2} \\
                  \vdots &\vdots& \ddots & \vdots \\
                  k_{1s} &k_{2s}& \cdots & -k_s \end{pmatrix}
             \begin{pmatrix} \frac{1}{M_1} &     && \\ 
                                 & \frac{1}{M_2} && \\
                                 && \ddots & \\
                  && & \frac{1}{M_s} \end{pmatrix},
\]
where $M_i$ is molar mass. We can then write
\begin{equation} \label{eqn:reaction_system_entries}
d_{ij} =
  \begin{cases}
  k_{ji} \frac{M_i}{M_j}, \quad i\neq j, \\
  -k_{ij}, \quad i=j. \\
  \end{cases}
\end{equation}
We denote the rate constant of the $i$-th radionuclide
\[
  k_i=\sum_{j=1}^{s}k_{ij}=\sum_{j=1}^{s}b_{ij}k_i
\]
which is equal to a sum of partial rate constants $k_{ij}$. \hyperA{RadioactiveDecayProduct::branching-ratio}{Branching ratio} $b_{ij}\in(0,1)$ 
determines the distribution into different branches of the decay chain, holding $\sum_{j=1}^{s}b_{ij}=1$.

Notice, that physically it is not possible to create a chain loop, so in fact one can permutate the vector of 
concentrations and rearrange the matrix $D$ into a lower triangle matrix
\[
\mathbf{D} = \begin{pmatrix} d_{11} &  &  &  \\ 
                  d_{21} & d_{22} & &  \\
                  \vdots &\vdots& \ddots &  \\
                  d_{s1} &d_{s2}& \cdots & d_{ss} \end{pmatrix}.
\]
However, we do not do this and we do not search the reactions for chain loops.

The system of first order differential equations with constant coefficients is solved using one of the
implemented linear ODE solvers, described in section \ref{sec:num_slode}.


\subsection{First Order Reaction}
\label{sec:first_order_reaction}
First order kinetic reaction is another process that can take part in the reaction term. Similarly to the
radioactive decay, it is connected to transport by operator splitting method and can run in all the possible
phases, see figure \ref{fig:reaction_term}.

Currently, reactions with single reactant and multiple products (decays) are available in the software.
The mathematical description is the same as for the radioactive decay, it only uses 
\hyperA{Reaction::reaction-rate}{kinetic reaction rate} coefficient $k$ in the input instead of half life.







% OLD
% The software suppports linear chemical reactions in the transport operator splitting method. 
% The linear chemical reactions (we will recall them only as 'reactions' in this section) can desribe
% \begin{itemize}
%   \item first order kinetic chemical reactions
%   \item radioactive decays, their chains and also complex chains with branches.
% \end{itemize}
% In the first case, the kinetics of a reaction is determined by a kinetic constant \hyperA{Substep::kinetic}{$k$}. 
% In the second case, the radioactive decay is determined by the half life of the reactant 
% \hyperA{Substep::half-life}{$t_{1/2}$}. Both notations
% %
% \[ A\xrightarrow{k} B, \qquad A\xrightarrow{t_{1/2}} B \]
% %
% express the same reaction and are governed by the same first order differential equation 
% %
% \[ \frac{\d c_A}{\d\tau} = -kc_A = - \frac{\ln 2}{t_{1/2}}\, c_A. \]
% %
% The relation between $k$ and $t_{1/2}$ is derived below
% \begin{eqnarray*}
%     \frac{\d c_A}{\d\tau} &=& -k c_A\\
%     \frac{\d c_A}{c_A} &=& -k \d\tau\\
%     \int\limits_{c_A^0}^{c_A^0/2}\frac{\d c_A}{c_A} &=& -k\int\limits_{0}^{t_{1/2}} 1\d\tau\\
%     \big[ \ln c_A\big]^{c_A^0/2}_{c_A^0} &=& -\big[k\tau\big]^{t_{1/2}}_{0}\\
%     \ln\frac{c_A^0}{2} - \ln c_A^0 &=& - k t_{1/2}\\
%     \ln 2 &=& k t_{1/2}\\
%     k &=& \frac{\ln 2}{t_{1/2}}.
% \end{eqnarray*}
% 
% 
% 
% Lets consider to have a narrow decay chain without branches. This kind of decay chain can be described by following equation
% \[
%  A\xrightarrow{t_{1/2,A}}B\xrightarrow{t_{1/2,B}}C\xrightarrow{t_{1/2,C}}D\xrightarrow{t_{1/2,D}}E,
% \]
% where letters $\{A,\ldots, E\}$ denotes isotopes contained in considered decay chain and ${t_{1/2},i},~i\in\{A,\ldots, E\}$ is a symbol for a half-life of $i$-th isotope.
% For a simulation of radioactive decay and first order reactions matrix multiplication based approach has been developed. It has been based on an arrangement of all the data to matrices. The matrix ${\bf C}^k$ contains the information about concentrations of all species ($s$) in all observed elements ($e$). The upper index $k$ denotes $k$-th time step. The matrix ${\bf C}^k$ has the dimension $e\times s~( rows \times columns)$.
% The transport simulation is realized by matrix multiplication 
% \[
%   {\bf T}\cdot{\bf C}^k = {\bf C}^{k+1},
% \]
% where ${\bf T}$ is a square, block-diagonal matrix, representing a set of algebraic equations constructed from a set of partial differential equations.
% When the simulation of the radioactive decay or the first order reaction is switched on, one step of
% simulation changes to 
% \[
%   {\bf T}\cdot{\bf C}^k\cdot{\bf R} = {\bf C}^{k+1},
% \]
% where ${\bf R}$ is a square matrix with the dimension $(s \times s)$ . It is much easier to construct and to use ${\bf R}$ , than to include chemical influence to the transport
% matrix ${\bf T}$ , because the matrix ${\bf R}$ has usually a simple structure and $s$ is much smaller than $e$. In the most simple case, when the order of identification numbers of isotopes in considered decay chain is the same as the order of identifiers of species transported by groundwater, then just two
% diagonals are engaged and the matrix R looks as follows:
% 
% \begin{tiny}\[
%    \begin{array}{l}
%     {\bf R} = \left(
%     \begin{array}{cccccc}
%      \left(\frac{1}{2}\right)^\frac{\Delta t}{t_{1/2,1}} & 1 - \left(\frac{1}{2}\right)^\frac{\Delta t}{t_{1/2,i}} & 0 & \hdots & \hdots & 0\\
%      0 & \left(\frac{1}{2}\right)^\frac{\Delta t}{t_{1/2,2}} & 1 - \left(\frac{1}{2}\right)^\frac{\Delta t}{t_{1/2,2}} & 0 & \ddots & 0 \\
%      \vdots & \ddots & \ddots & \ddots & \ddots & \vdots\\
%      0 & \ddots & 0 & \left(\frac{1}{2}\right)^\frac{\Delta t}{t_{1/2,n-2}} & 1 - \left(\frac{1}{2}\right)^\frac{\Delta t}{t_{1/2,n-2}} & 0 \\
%      0 & \hdots & \hdots & 0 & \left(\frac{1}{2}\right)^\frac{\Delta t}{t_{1/2,n-1}} & 1 - \left(\frac{1}{2}\right)^\frac{\Delta t}{t_{1/2,n-1}}\\
%      0 & \hdots & \hdots & 0 & 0 & 1
%     \end{array}\right)
%    \end{array}
% \]\end{tiny}
% 
% Every single $j$-th column, except the first one, includes the contribution $1 - \left(\frac{1}{2}\right)^\frac{\Delta t}{t_{1/2,j}},~j\in\{A,\ldots, E\}$ from $(j-1)$-th
% isotope with its half-life $t_{1/2,j-1}$. The term $\left(\frac{1}{2}\right)^\frac{\Delta t}{t_{1/2,j}}$ describes concentration decrease caused by the radioactive decay of $j$-th isotope itself. In general cases the matrix ${\bf R}$ can have much more complicated structure, especially when the considered decay chain has more branches.
% The implementation of the radioactive decay in Flow123D does not firmly include standard natural decay chain. Instead of that it is possible for a user to define his/her decay chain.
% 
% It is also possible to simulate decay chains with branches as picture \ref{pic:dec_branches} shows.
% 
% \begin{figure}[htb]
%  \centering
%  \includegraphics[width = 8cm]{\fig /decay_chain.png}
%  \caption{Decay chain with branches.}
%  \label{pic:dec_branches}
% \end{figure}
% 
% 
% When it comes to a simulation of first order reactions, the kinetic constant is given as an input. 
% The description of a kinetic chemical reaction has obviously two folowing forms
% \[
%   \begin{array}{l}
%     A\xrightarrow{k}B,\\
%     \frac{dc^A}{dt} = -k \cdot c^A.
%   \end{array}
% \]
% The first one description is a standard chemical one. The second equation describes temporal decrease in amount of concentrations of the specie $c^A$. The constant $k$ is so called kinetic constant and for the case of a first order reactions it is equal to so called reaction rate. The order of reaction with just one reactant is equal to the power of $c^A$ in partial diferential reaction.
% 
% For an inclusion of first order reaction into a reaction matrix a half-life needs to be computed from the corresponding kinetic constant $k$. The derivation follows
% \[
%   \begin{array}{l}
%     A\xrightarrow{k} B\\
%     \frac{dc^A}{d\tau} = -k\cdot c_A\\
%     \frac{dc^A}{c^A} = -k\cdot d\tau\\
%     \int\limits_{c^A_0/2}^{c^A_0}\frac{dc^A}{c^A} = -k\cdot\int\limits_{t_{1/2,A}}^{0} d\tau\\
%     \left[ ln c^A\right]_{c^A_0/2}^{c^A_0} = -[k\tau]_{t_{1/2,A}}^{0}\\
%     ln c^A_0 - \ln\frac{c^A_0}{2} = k\cdot t_{1/2,A}\\
% %     c^A (t) = c^A_0\cdot e^{-k\cdot t_{1/2,A}}\\
% %     {\bf substitution} \qquad c^A(t_{1/2,A}) = \frac{1}{2}\cdot c^A_0\\
% %     \frac{1}{2} c^A_0 = c^A_0\cdot e^{-k_1\cdot t_{1/2,A}}\\
%     \ln 2 = k \cdot t_{1/2,A}\\
%     t_{1/2,A} = \frac{ln 2}{k}
%   \end{array}                                                                                                                                                                                                                                                                                                            
% \]
% The matrix ${\bf R}$ is constructed in the same way as for the radioactive
% decay.
