\section{Fractures and dual porosity}

File: \texttt{06\_frac\_dualpor.yaml}

\subsection{Description}

This is a variant of \texttt{04\_frac\_diffusion.yaml}. Instead of
diffusion we consider advective transport with dual porosity.

\#\#Input Dual porosity substitutes blind fractures in this task. The
dual-porosity parameter \texttt{diffusion\_rate\_immobile} was
calibrated to the value 5.64742e-06 for identical results with the model
with the blind fractures. Other settings of transport are identical to
the diffusion model. \#\# Input Dual porosity substitutes dead-end
fractures in this task. The dual-porosity parameter
\texttt{diffusion\_rate\_immobile} was calibrated to the value
5.64742e-06 for identical results with the model with the dead-end
fractures. Other settings of transport are identical to the diffusion
model.

The dual porosity model is set by the following lines:

\begin{verbatim}
reaction_term: !DualPorosity
  input_fields:
    - region: rock
      init_conc_immobile: 0
    - region: flow_fractures
      diffusion_rate_immobile: 5.64742e-06
      porosity_immobile: 0.01
      init_conc_immobile: 0
    - region: blind_fractures
      init_conc_immobile: 0
\end{verbatim}

\#\#Results and comparison Results of calibration of the model with dual
porosity and model with flow in blind fractures (file
\texttt{06\_frac\_nodualpor.yaml}) is depicted in Figure
\ref{fig:calib}. \#\# Results and comparison Results of calibration of
the model with dual porosity and model with flow in dead-end fractures
(file \texttt{06\_frac\_nodualpor.yaml}) is depicted in Figure
\ref{fig:calib}.

\begin{figure}
\hypertarget{fig:calib}{%
\centering
\includegraphics{tutor_figures/06_mass_flux.pdf}
\caption{Results of calibration.}\label{fig:calib}
}
\end{figure}
