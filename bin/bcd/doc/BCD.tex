
\documentclass[a4paper]{article}

%\usepackage{czech}
\usepackage[cp1250]{inputenc}
\usepackage{fancyhdr}
\usepackage{graphicx}
\pagestyle{fancyplain}

\begin{document}

\Large
\begin{flushleft}
\textbf{BCD generator ini file format}


\normalsize
BCD generator version: 1.00 stable
\vspace{20pt}

Section: \textbf{\$Input}

Keys:
\end{flushleft}
\vspace{-20pt}
\small
\begin{tabbing}
=================================================\\
Output\_digits	\quad 	\= TYPE OF \quad \= DEFAULT	\quad 	\=DESCRIPTION\kill
KEY NAME	 			 				\> TYPE		 				\> DEFAULT	 				\>DESCRIPTION\\
												\> OF VALUE				\> VALUE						\>\\
=================================================\\
\textit{Mesh\_file}	\> \textbf{string}	\> NULL		\>Name of file containig definition\\
								 		\> 									\> 				\>of the mesh for the problem.\\
------------------------------------------------------------------------------------------------------------------\\
\textit{Ngh\_file} 	\> \textbf{string} 	\> NULL		\>Name of file describing topology\\
								 		\> 									\> 				\>of the mesh.\\
------------------------------------------------------------------------------------------------------------------\\
\textit{N\_subst}		\> \textbf{int}			\> 1			\>Number of substances.\\
------------------------------------------------------------------------------------------------------------------\\
\end{tabbing}

\normalsize

\begin{flushleft}
Section: \textbf{\$Output}

Keys:
\end{flushleft}
\vspace{-20pt}
\small
\begin{tabbing}
=================================================\\
Output\_digits	\quad 	\= TYPE OF \quad \= DEFAULT	\quad 	\=DESCRIPTION\kill
KEY NAME	 			 				\> TYPE		 				\> DEFAULT	 				\>DESCRIPTION\\
												\> OF VALUE				\> VALUE						\>\\
=================================================\\
\textit{FBC\_file} 	\> \textbf{string}	\> NULL		\>Flow boundary condition\\
										\> 									\> 				\>output file name.\\
------------------------------------------------------------------------------------------------------------------\\
\textit{TBC\_file} 	\> \textbf{string} 	\> NULL		\>Transport boundary condition\\
										\> 									\> 				\>output file name.\\
------------------------------------------------------------------------------------------------------------------\\
\textit{TIC\_file}	\> \textbf{string}	\> NULL		\>Transport initial condition\\
										\> 									\> 				\>output file name.\\
------------------------------------------------------------------------------------------------------------------\\
\textit{MTR\_file}	\> \textbf{string}	\> NULL		\>Materials output file name.\\
------------------------------------------------------------------------------------------------------------------\\
\textit{Output\_digits}	\> \textbf{int}	\> 12			\>Number of output digits in all output files.\\
------------------------------------------------------------------------------------------------------------------\\
\textit{Accuracy}		\> \textbf{double}	\> 1E-12	\>Maximal spacing between element\\ 
										\> 									\> 				\>boundary side nodes and closest defined\\ 
										\> 									\> 				\>general plane equation.\\
------------------------------------------------------------------------------------------------------------------\\
\textit{Write\_all\_BC}	\> \textbf{YES/NO}	\> NO	\>If set to YES, writes all boundary\\
												\> 									\> 		\>conditions (also all the Neumann\\
												\> 									\> 		\>homogenous conditions).\\
------------------------------------------------------------------------------------------------------------------\\
\end{tabbing}

\normalsize

\begin{flushleft}
Section: \textbf{\$FBC} (flow boundary condition)


\textit{con\_def\_type} \ $\left\langle type\_specific\_data\right\rangle$ \ \textit{bc\_type} \ $\left\langle bc\_specific\_data\right\rangle$ \ \textit{n\_tag} \ $\left\langle tags\right\rangle$
\end{flushleft}



\begin{flushleft}
Section: \textbf{\$TBC} (transport boundary condition)

\textit{con\_def\_type} \ $\left\langle type\_specific\_data\right\rangle$ \ \textit{ex} \ \textit{ey} \ \textit{ez} \ \textit{e1} \ \textit{n\_subst} \ \textit{subst\_list}

\end{flushleft}



\begin{flushleft}
Section: \textbf{\$TIC} (transport initial condition)

\textit{con\_def\_type} \ $\left\langle type\_specific\_data\right\rangle$ \ \textit{ex} \ \textit{ey} \ \textit{ez} \ \textit{e1} \ \textit{n\_subst} \ \textit{subst\_list}
\end{flushleft}



\begin{flushleft}
where:
\end{flushleft}
\vspace{-20pt}
\small
\begin{tabbing}
=================================================\\
Output\_digits	\qquad \quad	\= TYPE OF \quad\qquad  	\=DESCRIPTION\kill
PARAMETER	 			 				\> TYPE		 					\>DESCRIPTION\\
NAME										\> OF VALUE					\>\\
=================================================\\
\textit{con\_def\_type} 	\> \textbf{int}						\>5 types of condition definition [0,1,2,3,4]. See\\
													\> 												\>below for definitions of the types.\\
------------------------------------------------------------------------------------------------------------------\\
\textit{bc\_type}					\> \textbf{int}						\>Boundary condition type [1;2;3]. See below\\
													\> 												\>for definitions of the types.\\
------------------------------------------------------------------------------------------------------------------\\
\textit{n\_tag} 					\> \textbf{int}						\>Number of tags.\\
------------------------------------------------------------------------------------------------------------------\\
\textit{tags} 						\> \textbf{int[n\_tag]}	\>List of tags.\\
------------------------------------------------------------------------------------------------------------------\\
\textit{n\_subst}					\> \textbf{int}						\>Number of substances.\\
------------------------------------------------------------------------------------------------------------------\\
\textit{subst\_list}			\> \textbf{int[n\_subst]}\>List of substances.\\
------------------------------------------------------------------------------------------------------------------\\
\end{tabbing}

\normalsize
\vspace{-20pt}

\begin{flushleft}
Each condition overwrites previous one. The last defined is applied!!
\end{flushleft}



\begin{flushleft}
\textbf{Types of boundary condition definitions} [\textit{con\_def\_type}]
\end{flushleft}
\vspace{-10pt}
Defines where the boundary condition is set (different \textit{type\_specific\_data}).
\vspace{-5pt}

\begin{table}[ht]

		\begin{tabular}{||c|l||}
				\hline
				Type				&	Description where the boundary condition is set\\\hline\hline
				$0$					&	On all boundaries of area\\\hline
				$1$					&	On listed element boundaries\\\hline
				$2$					&	On element boundaries which nodes are closer to defined equation\\ 
										& than a value of \textit{Accuracy} key\\\hline
		  	$3$					&	On element boundaries which material ID corresponds to listed ID\\\hline
		  	$4$					&	Both 2 and 3 types must be fulfilled together\\\hline
		\end{tabular}
\end{table}

\begin{flushleft}
\textbf{Types of initial condition definitions} [\textit{con\_def\_type}]
\end{flushleft}

\vspace{-10pt}
Defines where the initial condition is set (different \textit{type\_specific\_data}).
\vspace{-5pt}

\begin{table}[ht]

		\begin{tabular}{||c|l||}
				\hline
				Type				&	Description where the initial condition is set\\\hline\hline
				$0$					&	On all elements of area\\\hline
				$1$					&	On listed elements\\\hline
				$2$					&	On elements which centre of gravity are in the half-space\\ 
										& defined with equation\\\hline   %than a value of \textit{Accuracy} key\\\hline
		  	$3$					&	On elements which material ID corresponds to listed ID\\\hline
		  	$4$					&	Both 2 and 3 types must be fulfilled together\\\hline
		\end{tabular}
\end{table}

\newpage

\begin{flushleft}
\textbf{Type specific data} $\left\langle type\_specific\_data\right\rangle$
\end{flushleft}

\textbf{type = 0}: \quad
 $\left\langle type\_specific\_data\right\rangle$ = NULL

\textbf{type = 1}: \quad
 $\left\langle type\_specific\_data\right\rangle$ = \textit{n\_elem elm\_1\_id elm\_2\_id ... elm\_n\_id}

\textbf{type = 2}: \quad
 $\left\langle type\_specific\_data\right\rangle$ = \textit{bx by bz b1}

\textbf{type = 3}: \quad
 $\left\langle type\_specific\_data\right\rangle$ = \textit{mid}

\textbf{type = 4}: \quad
 $\left\langle type\_specific\_data\right\rangle$ = \textit{bx by bz b1 mid}

%\newpage

\begin{flushleft}
\textbf{Types of boundary conditions} [\textit{bc\_type}]
\end{flushleft}

\vspace{-10pt}
Defines where the initial condition is set (different \textit{bc\_specific\_data}).
\vspace{-5pt}

\begin{table}[ht]

		\begin{tabular}{||c|l||}
				\hline
				Type				&	Description\\\hline\hline
				$1$					&	Boundary condition of the Dirichlet's type\\\hline
				$2$					&	Boundary condition of the Neumann's type\\\hline
		  	$3$					&	Boundary condition of the Newton's type\\\hline
		\end{tabular}
\end{table}


\begin{flushleft}
\textbf{BC specific data} $\left\langle bc\_specific\_data\right\rangle$
\end{flushleft}

\textbf{bc\_type = 1}: \quad
 $\left\langle bc\_specific\_data\right\rangle$ = \textit{ex ey ez e1}

\textbf{bc\_type = 2}: \quad
 $\left\langle bc\_specific\_data\right\rangle$ = \textit{ex ey ez e1}

\textbf{bc\_type = 3}: \quad
 $\left\langle bc\_specific\_data\right\rangle$ = \textit{sigma ex ey ez e1}


\begin{flushleft}
where:
\end{flushleft}
\vspace{-20pt}
\small
\begin{tabbing}
=================================================\\
Output\_digits	\qquad \quad	\= TYPE OF \quad\qquad  	\=DESCRIPTION\kill
PARAMETER	 			 				\> TYPE		 					\>DESCRIPTION\\
NAME										\> OF VALUE					\>\\
=================================================\\
\textit{n\_elm} 			\> \textbf{int}						\>Number of elements.\\
------------------------------------------------------------------------------------------------------------------\\
\textit{elm\_n\_id} 	\> \textbf{int} 					\>ID of the n-element in the list.\\
------------------------------------------------------------------------------------------------------------------\\
\textit{mid}	 				\> \textbf{int}						\>Material ID.\\
------------------------------------------------------------------------------------------------------------------\\
\textit{sigma}	 			\> \textbf{double}				\>Newton's BC interchange coefficient.\\ % revidovat
------------------------------------------------------------------------------------------------------------------\\
\textit{bx by bz b1}	\> \textbf{double[4]}			\>Boundary equation coefficients. \\
                      \>                        \> $b(x,y,z) = bx \cdot x + by \cdot y + bz \cdot z + b1$ \\
------------------------------------------------------------------------------------------------------------------\\
\textit{ex ey ez e1}	\> \textbf{double[4]}			\>Equation of the boundary condition rule. \\
                      \>                        \> $e(x,y,z) = ex \cdot x + ey \cdot y + ez \cdot z + e1$ \\
------------------------------------------------------------------------------------------------------------------\\
\end{tabbing}

\normalsize



\end{document}